\documentclass[12pt]{book}
\usepackage{version}
\usepackage[T1]{fontenc}
\usepackage{longtable}
\usepackage{graphicx}
\usepackage{index}
\usepackage[colorlinks=true,hyperindex=true]{hyperref}
\makeindex

\begin{document}
\author{\href{http://www.freecol.org/team-and-credits.html}{The FreeCol Team}}
\title{FreeCol Documentation\\Developer Guide for Version \fcversion}
\maketitle{}

\tableofcontents
\newpage


\hypertarget{How to become a FreeCol developer}
            {\chapter{How to become a FreeCol developer}}


\hypertarget{The goal of our project}{\section{The goal of our project}}

We are aiming towards making a clone of the old computer
classic "Sid Meier's Colonization".

New features should in general not be added before we reach
FreeCol 1.0.0, unless they are implemented as optional features
using the class "GameOptions".

(Note that if you add a game option, add a default value setting in
FreeColServer.fixGameOptions so that use of the option does not break
old games.  The same applies to client-options,
in ClientOptions.fixClientOptions.)

The big exception to this rule is the our client-server model that
will allow players from all over the world to compete in a game
of FreeCol.

Read more \href{http://www.freecol.org/about.html}{here}.


\hypertarget{SourceForge project site}{\section{SourceForge project site}}

You should visit and get familiar with our project site at
\href{https://sourceforge.net/projects/freecol/}{SourceForge}.

This site contains trackers for bugs and features requests,
a task manager and lots of other important services.

We expect developers to use this site regularly when making
changes to the codebase.


\hypertarget{How to get tasks}{\section{How to get tasks}}

You may find available tasks in the bug and feature request trackers
--- just grab any task you are planning to do (in the immediate
future) by posting a comment telling that you intend to do this task.
Please create a new bug report/feature request for tasks that are not
listed here (and before you start working on them).  Please remember
to post a comment when you are done, or if you are unable to complete
the work.

Major changes to the code should be discussed on the developer's
mailing list before they are implemented.  This is to ensure that your
work will not be in vain if somebody else knows a better way of doing
it.
% [Currently moot, we have no working roadmap] 
% It is also a good idea to discuss changes that is not directly
% related to the next release on our
% \href{http://www.freecol.org/index.php?section=18}{roadmap}.


\hypertarget{How to use the trackers}{\section{How to use the trackers}}

If you are a full developer (i.e. with write privileges), this is how
a bug tracker item should be updated while you are working:

\begin{enumerate}
\item Assign yourself to the tracker item before you start working on it.

\item Please verify that no duplicate entry has been posted.

\begin{itemize}
\item If you can find a duplicate and the bug has \emph{not} been
  fixed, please set its status to ``Closed-Duplicate'', its milestone
  to ``Unspecified'', and post a comment with the ID of the other
  tracker item (add a comment to the other tracker item as well).

\item If you can find a duplicate and the bug has been fixed, use the
  ``Closed-out-of-date'' status to close the tracker item.
\end{itemize}

\item Set the item's status to ``Open-Needs-Info'' if you require
  input from the person originally submitting the item.

\item Set the item's status to ``Open-WWC1D'' if the fix requires
  determining what Col1 did (``What Would Col1 Do?'').

\item If you are unable to complete the item: assign it back to
  ``None'' and make a comment describing any problems relevant for
  another developer.  The milestone of open bugs should be ``Current''.

\item If you successfully commit a fix for a bug, set the milestone to
  ``Fixed-trunk'', the status to ``Closed'' if you are certain of the
  fix, or ``Pending-Fixed'' if there is some uncertainty and/or
  further comment is welcome.  Please also write a comment telling
  that the work is done, and it is helpful to refer to any commit/s
  where relevant changes occurred.
\end{enumerate}


This is how a feature request item should be updated while you are
working:\emph{(Needs updating since the sourceforge migration)}

\begin{enumerate}
\item Assign yourself to the tracker item before you start working on it.

\item Please verify that no duplicate entry has been posted.

\begin{itemize}
\item If you can find a duplicate and the feature has NOT been
  implemented: Please use the group "Duplicated", set the status to
  "Closed" and post a comment with the ID of the other tracker item
  (add a comment to the other tracker item as well).

\item If you can find a duplicate and the feature has been
  implemented: Use the group "Out of date" and close the tracker item.
\end{itemize}

\item Set the item's status to pending if you require input from the person
   originally submitting the item.

\item Set the group to "Accepted" if you decide that this feature
   request should be added before the next release (please discuss on
   the mailing list if there are any reasons not to include this
   feature).

\item If you are unable to complete the item (or you think someone else
   should be implementing it): Assign it to "None" and make a comment
   describing any issues relevant for another developer.

\item After you have completed the work: Set the group to "Added"
   and the status to "Closed". Please also write a comment telling that
   the work is done.
\end{enumerate}

You can use any suitable canned response instead of writing a comment.


\hypertarget{Mailing lists}{\section{Mailing lists}}

Our primary means of communication is the developers mailing list:
freecol-developers@lists.sourceforge.net

You can (and should) subscribe to this list
\href{http://lists.sourceforge.net/lists/listinfo/freecol-developers}{here}.


\hypertarget{Git}{\section{Git}}

Git is the tool we are using to manage the changes within our source
code tree.  This system makes it possible for all developers to have
their own full copy of the project, and supports synchronization
between the central version of the code (`the repository') and the
local copies.  Git also makes it possible to undo changes that were
previously committed to the repository.

\href{http://www.freecol.org/documentation/git.html}{This page}
describes how you can start using Git and get a working copy of the
code (without commit privileges).

You can use \verb+git pull+ for updating an existing working
copy.  Changes can only be applied by those who have write-access, so
you may need to either send the changes to the developer mailing list
or use the patch tracking system.


\hypertarget{Compiling the code}{\section{Compiling the code}}

We use the \textit{Apache Ant} build system for compiling the
game. You can get a copy of this program
\href{http://ant.apache.org}{here}. After it has been installed, you
simply type \verb+ant+ in the top directory of your "working copy" in
order to compile the game.  The file \verb+FreeCol.jar+ will then be
generated, and you can start the game simply by writing:

\verb+java -Xmx256M -jar FreeCol.jar+


\hypertarget{Using an IDE}{\section{Using an IDE}}

Most FreeCol developers don't seem to use an IDE, so there is no
``official'' setup available. However, in the config folder, you can
find contributed configuration files for both Netbeans and Eclipse.
The following information has also been contributed by players that do
use an IDE.

\hypertarget{Using Eclipse}{\subsection{Using Eclipse (thanks to ``nobody'')}}

\emph{This section is out of date since we migrated from svn to git.
Leaving as-is for now in the hope the procedure is similar.}

Since I'm quite a fan of the Eclipse IDE, I thought I would share my
experience with building FreeCol in Eclipse on the Windows platform.

I assume that you have installed JDK, Eclipse and SVN (both Eclipse
plug-in and stand alone client). Make sure that your path environment
variable contains both the JDK and SVN client directories.

First, add a new repository location in Eclipse in the SVN
Repositories view, for the
\href{https://svn.freecol.org/svnroot/freecol/freecol/}{FreeCol
  repository}. Leave all the other settings unchanged and click
Finish.

Select either 'trunk' or the branch you want to build, right-click on
it and select 'Find/Check out as...'

In the Check Out dialog, make sure the option 'Check out as a project
configured using the New Project Wizard' is selected, and click
Finish.

Select 'Java Project' in the 'Java' category, and click Next. Name
your project (FreeCol is an obvious choice). Leave all the other
options as is, and click Finish.

Eclipse starts to copy all the files from the repository. Depending on
the server and your connection, this may take from 1-10 minutes. Get a
cup of coffee or a glass of cold milk while you wait.

After downloading has finished, you should see your new project in the
Project Explorer. Right-click on the project and select 'Configure
Build Path...' from the 'Build Path' sub-menu.

First off, let's make sure that Eclipse has detected the source file
folder 'src'. Select the 'Source' tab, and make sure there is an entry
with the name [project name]/src. If not, add it. Don't close the
window, as we need to make other changes here.

Next, we have to add the external jar files to the project, so Eclipse
can properly verify the code. Select the 'Libraries' tab, and click
the 'Add JARs...' button. Browse to the 'jars' subfolder in the
FreeCol project, and select all the jar-files by holding down the
CTRL-key. Click OK, and OK again.

Eclipse should now be able to properly build the project without any
errors. If not, fix it. However, we don't actually want Eclipse to do
this, since we instead want to use the Ant build file from the
repository.

Right-click on the project, and select 'Properties...' all the way at
the bottom of the menu. Select 'Builders' in the menu to the left. You
should now see one entry in the list, named 'Java Builder'. This is
the default, built-in java builder in Eclipse. Click 'New...' to
create our Ant builder instead. Select 'Ant Builder' from the list and
click OK.

In the configuration dialog, click the 'Browse Workspace...' button
the 'Buildfile' section. Click on the FreeCol project, and select
'build.xml' from the list on the right. Click OK. Click OK again, and
the Ant builder is created. You can keep both builders active at the
same time, but if you want to save processing power, you can uncheck
the 'Java Builder'. Eclipse will warn you about doing this, but don't
be alarmed, you can always turn it on again.

Click OK. If you have activated Automatic building in Eclipse, Ant
should start building the project right away. Possible errors could
be, that Ant cannot access either the java compiler or a stand-alone
svn client. In either of these cases, make sure you added the right
directories to your path environment variable.

If the build went succesfull; congratulations. Open the project folder
in the file system, and you will see 'FreeCol.jar' in the root
folder. Since this is an executable jar file, you can double click it
and launch the game right away. Enjoy.


\hypertarget{Using Netbeans}{\subsection{Using Netbeans (thanks to ``xsainnz'')}}

\begin{itemize}
\item In Netbeans, Select File > New Project
\item New Project Window
\begin{itemize}
\item Select Java Category, Java Free-Form Project
\end{itemize}
\item Name and Location Panel
\begin{itemize}
\item In the Location box, browse to where ever you put the source (.../freecol/)
\item It should auto detect the build file location, project name and folder
\end{itemize}
\item Build and Run Actions Panel
\begin{itemize}
\item Leave the settings as they are
\end{itemize}
\item Source Package Folders Panel
\begin{itemize}
\item Add the 'src' folder as Source packages and 'tests' as Test packages
\end{itemize}
\item Java Sources Panel
\begin{itemize}
\item Click 'Add JAR/Folder,
\item browse into the jars folder
\item select all of the jars
\item click open.
\end{itemize}
\item Click Finish
\end{itemize}

\hypertarget{Code documentation}{\section{Code documentation}}

Our primary code documentation is the Javadoc generated
documentation. You can convert this documentation to HTML by
typing \verb+ant javadoc+. The directory "javadoc" will then be
created and you can start watching the documentation by opening
"index.html" from that directory.

There is also some additional documentation
\href{http://www.freecol.org/documentation/}{here}.


\hypertarget{Quality of code}{\section{Quality of code}}

First of all, your code will be read and modified by several
different developers.  Therefore it is important to create a
block of JavaDoc documentation with all methods/classes/packages
you implement.

You should also spend more time thinking about the overall
structure than when you are working alone.

Please read the \href{http://java.sun.com/docs/codeconv/}{Java Code
  Conventions}.  This will only take about 15 minutes and will really
help you write beautiful code.

And one more thing.  Please configure your editor in such a way
that code indentations result in the insertion of 4 spaces, and avoid
using tabs.


\hypertarget{How to build a FreeCol release}{\chapter{How to make a FreeCol release}}

You will obviously need to have installed \texttt{ant} to do the
build, and \texttt{git} to make the commits, however this will be
normal for a developer.  To generate the online manual you will also
need \texttt{htlatex}, and \texttt{pdflatex} for the print manual.
You can avoid these requirements by setting the ant properties
\texttt{online.manual.is.up.to.date} and
\texttt{print.manual.is.up.to.date} respectively, however this is not
recommended for a release.

\begin{itemize}

\item Make sure that all relevant changes have been committed to the
  branch you are about to release.  If you plan to upload the JavaDoc
  (see below) make sure that \verb+ant javadoc+ is not finding errors
  before branching.

\item Merge translations from trunk if the release is not made from
  trunk, with \verb+ant merge-translations+.  Skip this step if the
  localization files are essentially the same as the ones in trunk (we
  try to ensure this).  Make sure they are up to date, however.

\item Lately we release from the git master and continue to work from
  there, so there is no urgent need to create a special release branch.

\item Change the \verb+FREECOL_VERSION+ constant in FreeCol.java to
  match the release version.

\item Likewise, change the value of the \verb+fcversion+ macro in
  \texttt{doc/version.sty} to the release version.

\item Start a clean compile, run all tests and verify the
  specification(s). You can do that by calling \verb+ant prepare-commit+.
 
\item Commit the changes.

\item Call \verb+ant dist+ in order to build all packages. You will be
  prompted for the version of this release. Alternatively, you can
  specify the version on the command line, by saying something like
\begin{verbatim}ant -Dfreecol.version=0.9.0-alpha2 dist
\end{verbatim}
  instead (replace 0.9.0-alpha2 with the correct version, of
  course). It might be necessary to increase the memory available for
  ant, for example by setting the environment variable
  \verb|ANT_OPTS="-Xms256m -Xmx256m"|.  Errors in language packs
  only apply to the installer and need not delay the overall release.

\item Install one of the generated packages and verify that you can play
  normally for at least five turns. Other good tests include loading a
  saved game and running the game in debug mode for a hundred turns or
  so. It might also be a good idea to compile the game from one of the
  source packages.

\item Upload the packages to \verb+sftp://frs.sourceforge.net/+ and ensure that
  all packages get uploaded. You can use a script similar to this:

\begin{verbatim}
#!/bin/sh

FREECOL_VERSION=0.9.0-alpha2
PREFIX=freecol-$FREECOL_VERSION
USERNAME=myUserName

cd dist

sftp $USERNAME,freecol@frs.sourceforge.net <<FILE
cd /home/frs/project/f/fr/freecol/freecol
mkdir $PREFIX
cd $PREFIX
put $PREFIX-installer.exe
put $PREFIX-installer.jar
put $PREFIX-mac.tar.bz2
put $PREFIX-src.zip
put $PREFIX.zip
exit
FILE
\end{verbatim}

\item Compile the feature list for the news announcement (use the
  tracker and/or run the previous version for comparison).  Include
  information on savegame compatibility with previous versions on all
  messages.

\item The \texttt{freecol.org} website is in transition to a new
  format.  For now, you will need to edit the \texttt{index.html},
  \texttt{news/index.html} and \texttt{download.html} files by hand.

%\item Update the release number in the "Version"-mambot at
%  \texttt{http://www.freecol.org/administrator} > Mambots > Site Mambots > Version.
%
%\item Publish/unpublish "Download Unstable Version" in the main menu
%  (Front Page Manager).  This page should only be displayed if the
%  unstable version is the latest release.
%
%\item Copy the previous release announcement (Content > All Content
%  Items), rename and update the copy for the current release, and
%  publish it to the website.  Retain the download button but fix the
%  version number on its label.
%
%\item Remove the download button from the previous release news item.

\item Post release messages on our mailing lists: developers,
  translators and users.  Beware that you may have to be subscribed to
  some lists to be able to post.The addresses are:
  \texttt{freecol-{developers,translators,users}@lists.sourceforge.net}

\item Post a release message to the FreeCol
  \href{https://sourceforge.net/p/freecol/discussion/141200/}{forum}.

\item Post a news item on the FreeCol project \href{https://sourceforge.net/p/freecol/news/}{news} page on sourceforge.

\item Regenerate the documentation (with \verb+ant manual+) and
  JavaDoc (with \verb+ant javadoc+) and update JavaDoc and User
  Documentation on \href{freecol.org}{freecol.org}. You can do that
  with a script like this:

\begin{verbatim}
#!/bin/sh

USERNAME=myUserName

cd doc

{
    echo cd /home/project-web/freecol/htdocs
    echo put FreeCol.pdf docs/
    echo put FreeCol.html docs/
    echo put images/* docs/images
    find javadoc/ -printf "put %p %p\n"
} | sftp $USERNAME,freecol@web.sf.net


\end{verbatim}

%% Omitting the following for now.  It will need to be revised if we
%% start doing unstable releases again.
%
%The following items can be omitted for unstable releases:
%
%\begin{itemize}
%
%\item Go to the bug tracker and create a new milestone (``Edit
%  Milestones'') called ``Fixed\_\emph{release}''
%  (e.g. ``Fixed\_0.10.2'').  Select the ``Fixed\_trunk'' group and do
%  a mass edit (the pencil icon in the top right of the bug tracker
%  list) and move all bugs to the new milestone.
%
%\item Do the corresponding action in other ticket categories that
%  have distinct release milestones.
%
%\item Create news items at \href{Linuxgames.com}{Linuxgames.com},
%  \href{linux-gamers.com}{linux-gamers.com} and
%  \href{happypenguin.org}{happypenguin.org}.

\end{itemize}

\hypertarget{Missing features}{\chapter{Missing features}}

We know that FreeCol does not yet emulate all features of the original
game.  However there are several features which are inevitably
different to Colonization due to FreeCol being a multiplayer game ---
interactions with other European players being the obvious example.
Similarly we do not attempt to exactly emulate the graphical look and
feel of Colonization, although no displayed information should be lost.

Otherwise, any missing feature from Colonization is considered to be a
bug.  Please report such omissions on the
\href{https://sourceforge.net/p/freecol/pending-features-for-freecol/}{pending
  features} tracker.

Since we do not have access to the source code of the original game,
we can only guess at the algorithms used.  This is particularly clear
in complex, detailed areas such as production and combat.  In some
cases, players have reverse-engineered the algorithm.  If you know of
some calcuation in FreeCol that differs from that of the original
game, please tell us about it.  There is an effort underway to
completely document Colonization's production amounts
\href{https://sourceforge.net/p/freecol/wiki/WWC1D%20-%20resource%20output%20v2/}{here}.


\hypertarget{Changing the Rules}{\chapter{Changing the Rules}}

We would like to make FreeCol configurable, so that the game engine
becomes capable of emulating many similar games. For this purpose,
we have made many of the game's features configurable.

At some point in the future, we will probably add a special rule set
editor, but at the moment, the most effective option is to edit the file
specification.xml directly. This file defines the abilities of units,
founding fathers, buildings, terrain types, goods and equipment, for
example. You can find this file in the \textit{data/freecol}
directory.  Try to avoid overriding the base Colonization-compatible
rules in \textit{data/classic}.  For a small self-contained rule
change, another option is to build a \emph{mod} --- see the Mods
section following.

This is still work in progress, however, and the schema for the rule
set certain to change again in the future. If you wish to develop your
own rule set, you will have to monitor FreeCol development closely.

This having been said, we are particularly interested in hearing about
problems caused by your changes to the rule set. Some dialogs might be
unable to display more types of goods than are currently defined, for
example. Or other dialogs might not recognize your new Minuteman unit
as an armed unit. Please help us improve FreeCol by telling us about
such problems.

If you have a working rule set that adds a new flavour to the game, we
will gladly distribute it along with our default rule set. If you have
ideas that can not currently be implemented, we will probably try to
remove these limitations.

If you try to modify the rule set, you are strongly encouraged to
check whether the result is still valid. You can do this by validating
the result with the command \verb$ant validate$.


\hypertarget{Modifiers and Abilities}{\section{Modifiers and Abilities}}

Most of the objects defined by the rule set can be customized via
modifiers and abilities.  Abilities are usually boolean values
(``true'' or ``false''). If the value is not explicitly stated, it
defaults to true. If an ability is not present, it defaults to
false. Modifiers define a bonus or penalty to be applied to a numeric
value, such as the number of goods produced by a unit. The modifier
may be an additive, multiplicative or a percentage modifier. Modifiers
default to ``identity'', which means they have no effect.

The code also checks that all abilities and modifiers it uses are
defined by the specification. Therefore, you must define all of them,
even if you do not use them. You can do this by setting their value to
the default value, e.g. ``false'' in the case of an ability, or ``0''
in the case of an additive modifier.

\newcommand{\ability}[1]{\index{#1}\index{Ability!#1}\hypertarget{#1}{\vspace{1em}\noindent\textbf{#1}}}
\newcommand{\modifier}[1]{\index{#1}\index{Modifier!#1}\hypertarget{#1}{\vspace{1em}\noindent\textbf{#1}}}
\newcommand{\affectsPlayer}{\\\textit{Affects: Player\\Provided by:
    Nation, Nation Type, Founding Father}}
\newcommand{\affectsUnit}{\\\textit{Affects: Unit\\Provided by:
    Nation, Nation Type, Founding Father, Unit Type, Equipment Type}}
\newcommand{\affectsBuilding}{\\\textit{Affects: Building\\Provided by:
    Building Type}}
\newcommand{\affectsColony}{\\\textit{Affects: Colony\\Provided by:
    Map}}
\newcommand{\affectsColonyTwo}{\\\textit{Affects: Colony\\Provided by:
    Building Type, Nation, Nation Type, Founding Father}}
\newcommand{\affectsTile}{\\\textit{Affects: Tile\\Provided by:
    Tile Type}}

TODO: This section is out of date (version 0.11.2).

\ability{model.ability.addTaxToBells}
\affectsPlayer

The player adds the current tax rate as a bonus to bells
production. The bonus is modified every time the tax increases or
decreases.

\ability{model.ability.alwaysOfferedPeace}
\affectsPlayer

The player is always offered peace in negotiations with AI players.

\ability{model.ability.ambushBonus}
\affectsUnit

The unit is granted an ambush bonus equal to the terrain's defence value.

\ability{model.ability.ambushPenalty}
\affectsUnit

The unit suffers an ambush penalty equal to the terrain's defence value.

\ability{model.ability.autoProduction}
\affectsBuilding

The building needs no units to produce goods, and will never produce
more goods than can be stored in the colony.

\ability{model.ability.automaticEquipment}
\affectsUnit

The unit automatically picks up equipment if attacked.

\ability{model.ability.automaticPromotion}
\affectsUnit

A unit that can be promoted will always be promoted when successful in
battle.

\ability{model.ability.betterForeignAffairsReport}
\affectsPlayer

The player is provided with more information about foreign powers.

\ability{model.ability.bombard}
\affectsUnit

The unit is able to bombard other units.

\ability{model.ability.bombardShips}
\affectsBuilding

The building has the ability to bombard enemy ships on adjacent tiles.

\ability{model.ability.bornInColony}
\affectsUnit

The unit can be born in a colony, provided that enough food is available.

\ability{model.ability.bornInIndianSettlement}
\affectsUnit

The unit can be born in an Indian settlement, provided that enough food is available.

\ability{model.ability.build}
\affectsBuilding

The building can build units or equipment.

\ability{model.ability.buildCustomHouse}
\affectsPlayer

The player can build custom houses.

\ability{model.ability.buildFactory}
\affectsPlayer

The player can build factories.

\ability{model.ability.canBeCaptured}
\affectsUnit

The unit can be captured. Land units that can not be captured are
destroyed, naval units that can not be captured are either sunk or
damaged.

\ability{model.ability.canBeEquipped}
\affectsUnit

The unit can be equipped.

\ability{model.ability.canNotRecruitUnit}
\affectsPlayer

The player can not recruit specified units.

\ability{model.ability.captureEquipment}
\affectsUnit

The unit can capture equipment from another unit.

\ability{model.ability.captureGoods}
\affectsUnit

The unit can capture goods from another unit.

\ability{model.ability.captureUnits}
\affectsUnit

The unit can capture enemy units.

\ability{model.ability.carryGoods}
\affectsUnit

The unit can transport goods.

\ability{model.ability.carryTreasure}
\affectsUnit

The unit can transport treasures, not treasure trains.

\ability{model.ability.carryUnits}
\affectsUnit

The unit can transport other units.

\ability{model.ability.convert}
\affectsUnit

The unit is a native convert.

\ability{model.ability.dressMissionary}
\affectsBuilding

The building can commission missionaries.

\ability{model.ability.electFoundingFather}
\affectsPlayer

The player can elect Founding Fathers.

\ability{model.ability.expertMissionary}
\affectsUnit

The unit is an expert missionary, but not necessarily commissioned.

\ability{model.ability.expertPioneer}
\affectsUnit

The unit is an expert pioneer, but not necessarily equipped with tools.

\ability{model.ability.expertScout}
\affectsUnit

The unit is an expert scout, but not necessarily equipped with horses.

\ability{model.ability.expertSoldier}
\affectsUnit

The unit is an expert soldier, but not necessarily equipped with muskets.

\ability{model.ability.expertsUseConnections}
\affectsPlayer

Experts working in factories can produce a small amount of goods even
if the raw materials are not available in the colony.

\ability{model.ability.export}
\affectsBuilding

The building can export goods to Europe directly.

\ability{model.ability.foundColony}
\affectsUnit

The unit can found new colonies.

\ability{model.ability.foundInLostCity}
\affectsUnit

The unit may be generated as the result of exploring a Lost City Rumour.

\ability{model.ability.hasPort}
\affectsColony

The colony has access to at least one water tile. This ability can not
be set by the specification, but it can be used as a required ability.

\ability{model.ability.ignoreEuropeanWars}
\affectsPlayer

The player will not be affected by the Monarch's declarations of war.

\ability{model.ability.improveTerrain}
\affectsUnit

The unit is able to improve terrain.

\ability{model.ability.independenceDeclared}
\affectsPlayer

The player has declared independence.

\ability{model.ability.mercenaryUnit}
\affectsUnit

The unit may be offered as a mercenary unit.

\ability{model.ability.missionary}
\affectsUnit

The unit is able to establish missions and incite unrest in native
settlements.

\ability{model.ability.moveToEurope}
\affectsTile

Units on the tile are able to move to Europe.

\ability{model.ability.multipleAttacks}
\affectsUnit

The unit can attack more than once.

\ability{model.ability.native}
\affectsUnit

The unit is a native unit.

\ability{model.ability.navalUnit}
\affectsUnit

The unit is a naval unit.

\ability{model.ability.pillageUnprotectedColony}
\affectsUnit

The unit is able to steal goods from and destroy buildings in an
unprotected colony.

\ability{model.ability.piracy}
\affectsUnit

The unit is a privateer.

\ability{model.ability.produceInWater}
\affectsBuilding

The building enables units to produce on water tiles.

\ability{model.ability.refUnit}
\affectsUnit

The unit can be part of the Royal Expeditionary Force.

\ability{model.ability.repairUnits}
\affectsBuilding

The building can repair units.

\ability{model.ability.royalExpeditionaryForce}
\affectsPlayer

The player is a Royal Expeditionary Force.

\ability{model.ability.rumoursAlwaysPositive}
\affectsPlayer

The player will always get positive results from exploring Lost City
Rumours.

\ability{model.ability.scoutForeignColony}
\affectsUnit

The unit can scout out foreign colonies.

\ability{model.ability.scoutIndianSettlement}
\affectsUnit

The unit can scout out native settlements.

\ability{model.ability.selectRecruit}
\affectsPlayer

The player can select a unit to recruit in Europe. This also applies
to units generated as a result of finding a Fountain of Youth.

\ability{model.ability.teach}
\affectsBuilding

The building enables experts to teach other units. However, the
building may place limits on the experience level of teachers.

\ability{model.ability.tradeWithForeignColonies}
\affectsPlayer

The player may trade goods in foreign colonies.

\ability{model.ability.undead}
\affectsUnit

The unit is an undead unit (used only in revenge mode).





\modifier{model.modifier.bombardBonus}
\affectsPlayer

The player's units are granted a bombard bonus when attacking.

\modifier{model.modifier.buildingPriceBonus}
\affectsPlayer

The player can build or buy buildings at a reduced price.

\modifier{model.modifier.defence}
\affectsUnit

The unit has a defence bonus or penalty.

\modifier{model.modifier.immigration}
\textit{Affects: Player\\Provided by: Goods Type}

Goods of this type contribute to the player's immigration points.

\modifier{model.modifier.landPaymentModifier}
\affectsPlayer

The player can buy Indian land at a reduced price.

\modifier{model.modifier.liberty}
\textit{Affects: Player\\Provided by: Goods Type}

Goods of this type contribute to the colony's and the owning player's
liberty points.

\modifier{model.modifier.lineOfSightBonus}
\affectsUnit

The unit has an increased line of sight.

\modifier{model.modifier.minimumColonySize}
\affectsColonyTwo

The population of the colony can not be voluntarily reduced below this
number. The modifier does not in any way affect a population reduction
due to starvation or other events.

\modifier{model.modifier.movementBonus}
\affectsUnit

The unit has an increased movement range.

\modifier{model.modifier.nativeAlarmModifier}
\affectsPlayer

The player generates less native alarm.

\modifier{model.modifier.nativeConvertBonus}
\affectsPlayer

The player has a greater chance of converting natives.

\modifier{model.modifier.nativeTreasureModifier}
\affectsPlayer

The player generates greater treasures when destroying native settlements.

\modifier{model.modifier.offence}
\affectsUnit

The unit has an offence bonus or penalty.

\modifier{model.modifier.religiousUnrestBonus}
\affectsPlayer

The player generates greater religious unrest in Europe.

\modifier{model.modifier.sailHighSeas}
\affectsUnit

The unit's travel time between Europe and the New World is reduced.

\modifier{model.modifier.tradeBonus}
\affectsPlayer

Prices in the player's market remain stable for longer.

\modifier{model.modifier.treasureTransportFee}
\affectsPlayer

The player pays a smaller fee for transporting treasures to Europe.

\modifier{model.modifier.warehouseStorage}
\affectsBuilding

The building increases the capacity of the warehouse.

\hypertarget{Mods}{\chapter{Mods}}

FreeCol packages a number of simple modifications to the FreeCol rules
and resources, generally known as \emph{mods}.  The standard mods live
in \texttt{.../data/mods}.  Users can add their own mods to a
\texttt{mods} directory under their main data directory.

A mod consists of a directory that includes at minimum a file
\texttt{mod.xml}, a file \texttt{FreeColMessages.properties}, and
other files that implement the required changes.  \texttt{mod.xml}
simply contains: \texttt{<mod id="}\emph{identifier}\texttt{" />}
where the identifier is some unique name (distinct from other existing
mods).  The \texttt{FreeColMessages.properties} file should contain at
minimum entries for \texttt{mod.}\emph{identifier}\texttt{.name} 
and \texttt{mod.}\emph{identifier}\texttt{.shortDescription}, so that
the mod selection dialog can display the mod correctly.  The messages
file should also contain other messages needed by the mod.

Many mods define extra resources.  If so, a
\texttt{resources.properties} file will be needed, and similarly files
for the resources involved.  For example, the ``example'' mod defines
a ``milkmaid'' unit, which needs an image, so the example mod
directory contains an image file for the milkmaid, and a reference to
this file in \texttt{resources.properties}.

Most mods change the specification.  This is done in a
\texttt{specification.xml} file.  This file is in the same general
format as the existing freecol rule sets, but does not attempt to
provide a comprehensive set.  The first non-comment line should be:
\texttt{<freecol-specification id="}\emph{identifier}\texttt{">}.
Note that mods typically do not specify which ruleset they extend.

When modifying a specification, expect additional elements you provide
to be applied additively, but attributes of an existing element are
cleared unless a special \texttt{preserve="true"} attribute is
present.  If you need to delete or redefine an element, respecify it
in context with just its \texttt{id} and \texttt{delete="true"}
attributes.  So for example, in the freecol ruleset we need to remove
the \texttt{model.modifier.minimumColonySize} modifier from the
stockade building, which is done as follows:

\begin{verbatim}
    <building-type id="model.building.stockade" preserve="true">
      <modifier id="model.modifier.minimumColonySize" delete="true" />
    </building-type>
\end{verbatim}

Note that not all elements take a \texttt{delete} tag yet.  You may
need to read or modify the source to be certain a mod will work.


\hypertarget{Resources}{\chapter{Resources}}

Various links pointing to more or less reliable information about the
original Colonization game:

\begin{itemize}
\item \href{http://strategywiki.org/wiki/Sid_Meier%27s_Colonization}
  {Strategy Wiki}
\item \href{http://www.colonization.biz/}{The Unofficial Microprose
  Colonization Home Page}
\item \href{http://dledgard0.tripod.com/FAQs/play_col_at_viceroy.htm}
  {Play Colonization at Viceroy level}
\item \href{http://www.colonizationfans.com/}{Colonization Fan Page}
\item \href{http://www.ibiblio.org/GameBytes/issue21/misc/colstrat.html}
  {Bill Cranston's Strategy guide}
\item \href{http://civilization.wikia.com/wiki/Colonization_tips}
  {Tomasz Wegrzanowski' Strategy Guide}, contains very valuable
  material on the number of bells required to elect a Founding Father,
  among other things
\end{itemize}


\printindex

\end{document}
